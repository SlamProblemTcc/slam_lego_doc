\chapter[Simple SLAM]{Simple SLAM: Auto-localização simplificada}

Neste capítulo são apresentados todos os componentes presentes no \textit{Simple SLAM}, incluindo a etapa de montagem do robô, a configuração do ambiente de programação e do ambiente de atuação do robô, o qual deve seguir algumas regras estabelecidas durante o Capítulo, para que a pesquisa possa ser refeita e analisada por outro pesquisador, buscando padronizar ao máximo as características da pesquisa.

Como o \textit{Simple SLAM} contempla um conjunto de técnicas de auto-localização e navegação, todas as técnicas utilizadas e analisadas estão presentes neste capítulo, possibilitando a análise parcial da solução como um todo. Como o objetivo principal da solução é a viabilização da utilização de técnicas de auto-localização em um contexto simplificado, como o presente no meio Educacional da robótica, durante este capítulo, as técnicas analisadas possuirão uma análise Educacional, apresentando diversas maneiras de trabalhar conceitos matemáticos durante a realização de atividades como a de Auto-localização.

\section{Contextualização}

	O principal incentivo do pesquisador em relação a esta proposta de trabalho faz referência às metodologias de ensino de matemática, física e programação, tanto no âmbito da graduação, quanto nos ensinos Básico, Fundamental e Médio. Assim como já foi apresentado durante o trabalho, mais especificamente na seção \ref{sec:robótica_educacional}, as metodologias de ensino geralmente aplicadas nos Centros Educacionais possuem uma característica de ensino passiva e ultrapassada. Desse modo, buscando garantir maior interesse dos alunos nos conteúdos apresentados, a Robótica Educacional se vê como uma ferramenta eficiente de ensino, como apresentam \cite{teachingWithRoboticKit}, \cite{construcionismoPapert}e \cite{roboticaEducativaEnsinoMedio}.

	Com isto em mente, o \textit{Simple SLAM} busca apresentar diversas técnicas de auto-localização que podem ser utilizadas como uma atividade Educacional, em um contexto de ensino. Além disso, o \textit{Simple SLAM} tem como objetivo a implementação de técnicas de auto-localização utilizadas no alto nível da robótica mundial em um contexto simplificado, utilizando apenas equipamentos disponíveis no Kit de Robótica da LEGO - NXT.

	De acordo com o apresentado ao longo do trabalho, a margem de erro associada aos movimentos e sensores do robô exige a utilização de técnicas que possam minimizá-la. O \textit{Simple SLAM} apresenta técnicas com este objetivo tanto para casos simples de auto-localização e navegação quanto para uma solução completa de auto-localização, na qual foi utilizado o filtro de Partículas, também conhecido como Filtro de Monte Carlo.

\section{Arquitetura do Robô}


\section{Montagem do Robô}

\section{Ambiente de Navegação}

\section{Configuração e Integração das Tecnologias}

\section{Arquitetura da Solução}

\section{Considerações Parciais}