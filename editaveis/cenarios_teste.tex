\chapter[Cenários de Teste]{Cenários de Teste}

O principal objetivo dos cenários de teste é apoiar a observação e análise sistemática de todas as técnicas utilizadas durante o trabalho, assim como as diferentes configurações do ambiente e do robô. Após a realização de todos os cenários de teste propostos, deve-se chegar a uma conclusão referente a viabilidade da utilização da técnica do Filtro de Partículas em um contexto simplificado da robótica. 

Os cenários de teste estão divididos em duas grandes áreas: \textit{Auto-localização} e \textit{Auto-localização e mapeamento de ambientes simultâneos}. Sabe-se que as técnicas utilizadas durante a Auto-localização também foram utilizadas durante a resolução do SLAM como um todo. Desse modo, buscou-se analisar separadamente cada técnica de auto-localização, minimizando ao máximo a margem de erros, para que depois as mesmas pudessem ser utilizadas na resolução do SLAM como um todo.

Nas seções abaixo estão apresentados cada cenário de teste, seguido de suas conclusões e de um exemplo de sua utilização em conteúdos educacionais, registrando sua capacidade educacional.

\section{Técnica 1 - Verificação de Perpendicularidade} % (fold)
\label{sec:tecnica1}

	Este cenário de teste apresenta a utilização de uma técnica simples, porém importante para a localização e, principalmente, para a correção de erros de navegação, advindos dos sensores odométricos e das derrapagens no solo, principalmente. Em um contexto de deslocamento, onde o robô sai do ponto \textit{A} ao ponto {B}, conhecendo suas coordenadas, a partir da utilização da geometria, é possível se orientar utilizando como base a perpendicularidade entre a frente do robô e a parede mais próxima.

	A Figura \ref{img:perpendicularidade} representa a lógica presente nesta técnica, onde o robô compara as distâncias obtidas enquanto rotaciona no próprio eixo. Ao verificar-se que a distância está aumentando, o robô verifica para o sentido oposto até que a distância aumente. Quando o mesmo obtem a menor distância, pode-se afirmar que o robô está perpendicular à parede.

	A partir da utilização desta técnica, é possível corrigir alguns erros, garantindo, por exemplo, que o robô faça um caminho em linha reta em direção a uma parede na sua frente.
% section section_name (end)

\section{Técnica 2 - Navegação paralela à parede} % (fold)
\label{sec:técnica_2_navegação_paralela_à_parede}
	
	Este cenário de teste tem o objetivo de garantir a navegação paralela à paredes, a qual pode ocorrer diversas vezes durante uma navegação no ambiente. Com o objetivo de possibilitar de maneira a realização desta técnica de maneira eficiente, será utilizado o terceiro motor do kit para controlar a direção do sensor de distância, viabilizando a navegação em linha reta.

	Esta viabilização é feita a partir da comparação das distâncias obtidas durante a navegação. O percurso ideal terá sempre a mesma medida obtida, já que o mesmo estará navegando paralelamente à parede. Caso a distância se altere, o robô identifica a mudança e corrige sua direção retornando ao valor correto. A Figura \ref{img:paralelamente} apresenta de maneira clara o funcionamento desta técnica.
% section técnica_2_navegação_paralela_à_parede (end)

\section{Técnica 3 - Correção de erros de rotação} % (fold)
\label{sec:técnica_3_correção_de_erros_de_rotação}
	
	Este cenário de teste tem como objetivo verificar a implementação de uma técnica de correção de erros em rotações. Para isso, o robô precisa se encontrar próximo a uma parede. Utilizando a técnica \ref{sec:tecnica1} para encontrar a posição perpendicular à parede, registrando a distância encontrada na posição perpendicular. Após este registro, o robô realiza uma rotação completa, de 360º e uma nova verificação da distância. A partir da diferente entre as distâncias, o robô corrige o erro obtido nas rotações. 

	Esta correção pode ser feita utilizando a variável \textit{fixErrorCount}, a qual é utilizada para ponderar seu valor na angulação de rotação desejada.
% section técnica_3_correção_de_erros_de_rotação (end)

\section{Técnica 4 - Direcionamento por bússola} % (fold)
\label{sec:técnica_4_direcionamento_por_bússola}
	
	Este cenário de teste tem como objetivo verificar as vantagens da utilização de uma bússola como fonte de informações sobre o direcionamento do robô. Após esta verificação espera-se obter uma comparação entre os resultados obtidos com e sem a utilização de uma bússola, já que os erros também se encontram na mesma.
% section técnica_4_direcionamento_por_bússola (end)