\chapter[Cenários de Teste]{Cenários de Teste}

O principal objetivo dos cenários de teste é apoiar a observação e análise sistemática de todas as técnicas utilizadas durante o trabalho, assim como as diferentes configurações do ambiente e do robô. Após a realização de todos os cenários de teste propostos, deve-se chegar a uma conclusão referente a viabilidade da utilização da técnica do Filtro de Partículas em um contexto simplificado da robótica. 

Os cenários de teste estão divididos em duas grandes áreas: \textit{Auto-localização} e \textit{Auto-localização e mapeamento de ambientes simultâneos}. Sabe-se que as técnicas utilizadas durante a Auto-localização também foram utilizadas durante a resolução do SLAM como um todo. Desse modo, buscou-se analisar separadamente cada técnica de auto-localização, minimizando ao máximo a margem de erros, para que depois as mesmas pudessem ser utilizadas na resolução do SLAM como um todo.

Nas seções abaixo estão apresentados cada cenário de teste, seguido de suas conclusões e de um exemplo de sua utilização em conteúdos educacionais, registrando sua capacidade educacional.

\section{Cenário de Teste 1} % (fold)
\label{sec:cenario1}

	Este cenário de teste apresenta a utilização de uma técnica simples, porém bastante importante para a localização e, principalmente, a correção de erros de navegação, advindos dos sensores odométricos e das derrapagens no solo, principalmente.

% section section_name (end)