\chapter[Análise dos Resultados]{Análise dos Resultados}

Este capítulo tem como objetivo levavantar e discutir os resultados obtidos durante a realização da pesquisa. Dentre estes resultados,
estão presentes as fontes de erros levantadas com a experiência da pesquisa, suas respectivas correções, dentro do possível, e a análise da
viabilidade da utilização do conceito de SLAM no contexto simplificado da Robótica Educacional. Desse modo, este capítulo está dividido
em duas seções: \ref{sec:fontes_de_erros} e \ref{sec:viabilidade}.

\section{Fontes de Erros}

Como foi apresentado ao longo do trabalho, o mundo da robótica está envolto em erros. Desse modo, deve-se identificar o motivo do erro
para o mesmo possa ser minimizado ao máximo, buscando garantir uma navegação adequada, no caso da robótica móvel.
Por este motivo, durante a realização desta pesquisa, buscou-se identificar o maior número de fontes de erros possível para que as mesmas fossem
atacadas, minimizando a margem de erro durante o processo de auto-localização.

Cada fonte de erro identificada foi listada, com o objetivo de descrevê-la e apresentar uma possível solução que minimize o erro
advindo desta fonte. Esta listagem se encontra nas sub-seções dispostas abaixo.

\subsection{Distância entre Rodas}

\subsection{Diâmetro da Roda}

\subsection{Deslizes entre a Roda e o Piso}

\subsection{Precisão dos Sensores Odométricos}

\subsection{Característica do Sensor de Distância}

\subsection{Colisões em Obstáculos}

\section{Análise de Viabilidade}
