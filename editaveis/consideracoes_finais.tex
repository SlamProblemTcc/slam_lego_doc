\chapter[Conclusão]{Conclusão}

Esta pesquisa buscou analisar o contexto de SLAM dentro dos limites da Robótica Educacional, com o objetivo de estudar a
viabilidade da utilização de técnicas renomadas na comunidade de robótica para solucionar o problema de SLAM. De acordo com o apresentado ao
longo da pesquisa, a comunidade de robótica busca solucionar o problema de SLAM utilizando diversas
técnicas e estratégias. Entretanto, boa parte da comunidade concorda na utilização de algumas estratégias, como o processamento remoto,
apresentado na tabela \ref{tab:resultadosRevisao}. Visto isso, este trabalho buscou seguir as estratégias mais utilizadas e reconhecidas
 para o contexto de pesquisa, como a utilização da arquitetura de processamento remoto e a utilização do Filtro de Partículas.

Como o contexto alvo da pesquisa faz referência a Robótica Educacional, buscou-se utilizar ferramentas simples,
presentes no kit de robótica Mindstorm NXT e disponíveis em ambientes educacionais, advindos de baixo investimento.
