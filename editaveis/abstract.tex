\begin{resumo}[Abstract]
 \begin{otherlanguage*}{english}
   
   Most of the researches related to robotics are mainly  focused on the mobility of the robot. That happens because of the necessity of navegation and self localization in the enviroment in most of the activities. For this purpose, the technique of SLAM ( Simultaneous-localization and mapping) has been developed in various contexts in all robotics community. Starting from the techinques of systematic review, this project has as initial objective to identify different techniques of SLAM that are used in several contexts. The techniques will be analysed to identify  the features that can be adapted to the context of simple robots, which are used in educational robotics. As the final objective, this project aims to solve the problem of SLAM in a simple form, using the Mindstorms NXT robotic kit. The solution of SLAM problem will be part of the relation between the techniques present in the mobile robotics and the important subjects of the educational context, as Mathematics and Physics.

   \vspace{\onelineskip}
 
   \noindent 
   \textbf{Key-words}: Auto-localization and environment mapping, SLAM problema, simple robots, educational robotic, Mindstorms NXT.
 \end{otherlanguage*}
\end{resumo}
