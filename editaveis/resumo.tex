\begin{resumo}
 
 	Grande parte das pesquisas relacionadas à robótica são focadas, principalmente, na mobilidade do robô. Isto ocorre pelo fato da necessidade, na maioria das atividades, da navegação e auto-localização no ambiente, por parte do robô. Com este objetivo, a técnica de SLAM (Auto-localização e mapeamento de ambientes simultâneos) vem sendo desenvolvida em diversos contextos por toda a comunidade de robótica do mundo. A partir da utilização da técnica de revisão sistemática, este trabalho tem como objetivo inicial levantar diferentes técnicas de SLAM utilizadas em diversos contextos, identificando características que possam ser adaptadas a um contexto simplificado, como o da Robótica Educacional. Como objetivo final, este trabalho buscará solucionar, de maneira simplista, o problema de SLAM utilizando o kit de robótica Mindstorm, da Lego. A solução do problema de SLAM fará parte do relacionamento entre técnicas presentes na robótica móvel e disciplinas importantes do contexto educacional, como matemática e física. 

 \vspace{\onelineskip}
    
 \noindent
 \textbf{Palavras-chaves}: Auto-localização, mapeamento de ambientes, SLAM, robótica educacional, NXT Lego.
\end{resumo}
