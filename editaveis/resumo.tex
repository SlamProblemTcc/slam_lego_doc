\begin{resumo}

 	Grande parte das pesquisas relacionadas à robótica é focada, principalmente, na mobilidade do robô. Isto ocorre pela necessidade,
  na maioria das atividades, da navegação e auto-localização no ambiente. Com este objetivo, a técnica de SLAM (Auto-
  localização e mapeamento simultâneos de ambientes) vem sendo implementada em diversos contextos por toda a comunidade de robótica.
  Esta pesquisa buscou analisar técnicas renomadas de auto-localização no contexto da robótica mundial, a partir da execução de uma
  revisão sistemática sobre o tema, selecionando a técnica do Filtro de Partículas para adaptação e implementação no contexto limitado
  da Robótica Educacional. Durante as etapas de implementação e análise dos resultados, a pesquisa busca documentar de maneira
  clara e objetiva os procedimentos realizados, garantindo a possibilidade da execução dos procedimentos por interessados no assunto.
  Além da aplicação no contexto educacional, deve-se ressaltar que esta pesquisa faz referência a utilização de robôs simples
  no processo de auto-localização, o que abrange sua utilização também em contextos reais, porém com limitações de \textit{hardware}.

 \vspace{\onelineskip}

 \noindent
 \textbf{Palavras-chaves}: Auto-localização, Filtro de Partículas, SLAM, robôs simples, Mindstorms NXT.
\end{resumo}
