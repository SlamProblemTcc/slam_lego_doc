\begin{resumo}

 	Grande parte das pesquisas relacionadas à robótica é focada, principalmente, na mobilidade do robô. Isto ocorre pela necessidade, na maioria das atividades, da
navegação e auto-localização no ambiente. Com este objetivo, a técnica de SLAM (Auto-
localização e mapeamento simultâneos de ambientes) vem sendo desenvolvida em diversos
contextos por toda a comunidade de robótica. A partir da utilização da técnica
de revisão sistemática, este trabalho tem como objetivo inicial levantar técnicas
de SLAM utilizadas em variados contextos. As técnicas serão analisadas para identificar características que possam ser adaptadas ao contexto de robôs simples, presente na Robótica Educacional.
Como objetivo final, este trabalho busca analisar a viabilidade da solução do problema de SLAM utilizando o kit de robótica Mindstorms NXT, da Lego, dado o contexto limitado
presente no kit em questão.
A solução do problema de SLAM engloba o relacionamento entre técnicas presentes na robótica móvel e disciplinas importantes do contexto educacional, como matemática e física.

 \vspace{\onelineskip}

 \noindent
 \textbf{Palavras-chaves}: Auto-localização, Filtro de Partículas, SLAM, robótica educacional, Mindstorms NXT.
\end{resumo}
